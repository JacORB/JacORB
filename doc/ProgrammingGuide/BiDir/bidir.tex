
BiDirectional GIOP has its main use in configurations involving callbacks with
applets or firewalls where it sometimes isn't possible to open a direct
connection to the desired target. As a small example, imagine that you want to
monitor the activities of a server via an applet. This would normally be done
via a callback object that the applet registers at the server, so the applet
doesn't have to poll the server for events. To accomplish this without
BiDirectional GIOP, the server would have to open a new connection to the
client which will not work because applets usually are not allowed to act as
servers, i.e. open ServerSockets. At this point BiDirectional GIOP can help
because it allows to reuse the connection the applet opened to the server for
GIOP requests from the server to the applet (which isn't allowed in
``standard'' GIOP).

\section{Setting up Bidirectional GIOP}

Setting up BiDirectional GIOP consists of two steps:
\begin{enumerate}
\item Setting an ORBInitializer property and  creating the BiDir policy
\item Adding this policy to the servant's POA.
\end{enumerate}

\subsection{Setting the ORBInitializer property}
\label{sec:setting-up-bidir-orbinitializer}

The first thing that is necessary for BiDirectional GIOP to be available is
the presence of the following property, which can be added by the usual ways
(see chapter \ref{ch:configuration}):

\begin{verbatim}
   org.omg.PortableInterceptor.ORBInitializerClass.bidir_init=
       org.jacorb.orb.giop.BiDirConnectionInitializer
\end{verbatim}

If this property is present on ORB startup, the corresponding policy factory
and interceptors will be loaded.


\subsection{Creating the BiDir Policy}
Creating the necessary BiDir Policy is done via a policy factory hidden in the
ORB.

\begin{verbatim}
import org.omg.BiDirPolicy.*;
import org.omg.CORBA.*;

[...]

Any any = orb.create_any();
BidirectionalPolicyValueHelper.insert( any, BOTH.value );

Policy p  = orb.create_policy( BIDIRECTIONAL_POLICY_TYPE.value,
                               any );
\end{verbatim}

The value of the new policy is passed to the factory inside of an any. The ORB
is the told to create a policy of the specified type with the specified
value. The newly created policy is then used to create a user POA. Please note
that if {\em any} POA of has this policy set, {\em all} connections will be
enabled for BiDirectional GIOP, that is even those targeted at object of POAs
that don't have this policy set. For the full source code, please have a look
at the bidir demo in the {\tt demo} directory.


\section{Verifying that BiDirectional GIOP is used}
From inside of your application, it is impossible to tell whether requests
arrived over a unidirectional or BiDirectional connection. Therefore, to check
if connections are used in both directions, you can either use a network
monitoring tool or take a look at JacORBs output to tell you if your server
created a new connection to the client, or if the existing one is being
reused.

If the debug level is set to 2 or larger, the following output on the server
side will tell you that a connection is being reused:

\begin{verbatim}
[ ConnectionManager: found conn to target <my IP>:<my port> ]
\end{verbatim}

If, on the other hand, the connection is not being reused, the client will
show the following output:
\begin{verbatim}
[ Opened new server-side TCP/IP transport to <my host>:<my port> ]
\end{verbatim}


\section{TAO interoperability}
There is one problem that may prevent TAO and JacORB to interoperate using
BiDirectional GIOP: If JacORB uses IP addresses as host names (JacORBs
default) and TAO uses DNS names as host names (TAOs default on non Microsoft
Windows platforms), connections from JacORB clients to TAO servers will not be
reused. If, on the other hand, both use the same ``format'' for host addresses,
interoperability will be successful. There are two ways to solve this problem:
\begin{enumerate}
\item Use {\tt ``-ORBdotteddecimaladdresses 1''} as an command line argument
  to the TAO server.
\item Configure JacORB to use DNS names (See the configuration guide).
\end{enumerate}


%%% Local Variables:
%%% mode: latex
%%% TeX-master: "../ProgrammingGuide"
%%% End:
