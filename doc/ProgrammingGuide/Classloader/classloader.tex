%
% $Id: classloader.tex,v 1.1 2009-06-05 07:45:40 rnc Exp $
%

This chapter explains some of the problems that may be encountered with classpath and classloaders.

\section{Running applications}

By default JacORB is shipped with runtime scripts to simplify running
an application.  These scripts use the Java Endorsed Standards
Override Mechanism in order to ensure that the JacORB implementation
classes and the supplied OMG classes are found in preference to any
bundled within the JVM. This mechanism is documented here
http://java.sun.com/j2se/1.5.0/docs/guide/standards


The mechanism utilises the Xbootclasspath to place the classes first. If
this is not used then the Sun OMG classes may be found first. The issue
that may be encountered here is if JacORB is released with newer versions 
of the OMG classes than is distributed within the JVM. Therefore the JacORB 
classes should be used in preference.


\subsection{ORBSingleton}
Unlike an ORB.init(args,props) where a developer may pass arguments initialising
an ORBSingleton with ORB.init() does not. This means that unless the developer has
either

\begin{itemize}
\item Started the JVM supplying ORBSingletonClass and ORBClass properties
\item Overridden System properties prior to calling ORBInit with ORBSingletonClass and ORBClass properties
\end{itemize}

 the OMG ORB class will initialise the wrong ORBSingleton \textbf{if} endorsed 
directories are \textbf{not} being used. If endorsed directories are being used
the JacORB OMG ORB class will automatically load the correct Singleton.


\section{Interaction with Classloaders}

The endorsed directory mechanism means that the JacORB classes will be loaded
into the bootstrap classloader. If the developer has chosen to instantiate their
own child classloader and load the JacORB classes within this (e.g. via URLClassLoader 
downloading the classes over the network) several problems may be encountered:

\subsubsection{Garbage Collection}
The Sun JVM will load its OMG ORB classes in preference to those within the child
classloader. This means that it will retain a static link to the JacORB ORBSingleton
implementation within the child classloader. Therefore the classes cannot be fully garbage
collected once the classloader is disposed of.

\subsubsection{Class Conflict}
As described above the Sun OMG ORB class matains a static ORBSingleton reference. If a second
class loader is instantiated, as a ORBSingleton already exists in the parent bootclassloader
it will not be created. However when the JacORB code checks that

{\tt ORB.init () instanceof org.jacorb.orb.ORBSingleton }

it will fail. This is because the ORBSingleton class created in the first classloader is different
to the ORBSingleton class created in the second classloader. This behaviour is documented within the
Java Language Specification here http://java.sun.com/docs/books/jls/third\_edition/html/execution.html\#12.1.1
and a paper describing the behaviour may be found here http://www.tedneward.com/files/Papers/JavaStatics/JavaStatics.pdf


\subsubsection{Solving the Problem}
The above problem occurs as java.net.URLClassLoader uses the parent-first class-loader delegation model.
To solve the issue, the simplest and most effective solution is to use child-first class-loader delegation
model. An example of this may be found here http://www.qos.ch/logging/classloader.jsp

This model ensures that parent delegation occurs only \textbf{after} an unsuccessful attempt to load a class from
the child. Therefore the org.omg.CORBA.* classes supplied with JacORB would be found and used in preference to
the OMG classes supplied by Sun in the bootclassloader. The ORBSingleton would be created entirely within the child
classloader with no external references. This means the second classloader would also create its own, entirely 
isolated Singleton class.  
