
Until CORBA 2.3, objects could only be passed using reference
semantics: there was no way to specify that object state should be
copied along with an object reference. A further restriction of the
earlier CORBA versions was that all non-object types (structs, unions,
sequences, etc.) were {\em values}, so you could not use, e.g. a
reference-to-struct to construct a graph of structure values that
contained shared nodes. Finally, there was no inheritance between
structs.

All these shortcomings are addressed by the {\em objects-by-value} (OBV)
chapters of the CORBA specification: the addition of stateful
value types supports copy semantics for objects and inheritance for
structs, boxed value types introduce reference semantics for base
types, and abstract interfaces determine whether an argument is sent
by-value or by-reference by the argument's runtime type. The
introduction of OBV into CORBA presented a major shift in the CORBA
philosophy, which had been to strictly avoid any dependence on
implementation details (state, in particular). It also added a
considerable amount of marshaling complexity and interoperability
problems. (As a personal note: Even in CORBA 2.6, the OBV marshaling
sections are still not particularly precise...)

JacORB 2.0 implements most of the OBV specification.  Boxed value
types and regular value types work as prescribed in the standard
(including value type inheritance, recursive value types, and
factories).  Still missing in the current implementation is run-time
support for abstract value types (although the compiler does accept
the corresponding IDL syntax), and the marshaling of truncatable
value types does not yet meet all the standard's requirements (and
should thus be called ``beta'').

\section{Example}

To illustrate the use of various kinds of value types, here's an
example which is also part of the demo programs in the JacORB
distribution.  The demo shows the use of boxed value types and a
recursive stateful value type. Here's the IDL definition from {\tt
demo/value/server.idl}:

\begin{verbatim}
module demo {
  module value {

     valuetype boxedLong   long;
     valuetype boxedString string;

     valuetype Node {
        public long id;
        public Node next;
     };

     interface ValueServer {
        string receive_long   (in boxedLong p1, in boxedLong p2);
        string receive_string (in boxedString s1, in boxedString s2);
        string receive_list   (in Node node);
     };
  };
};
\end{verbatim}

From the definition of the  boxed value type {\tt boxedLong} and
{\tt boxedString}, the IDL generates the following Java class, which
is simply a holder for the long value. No mapped class is generated
for the boxed string value type.

\begin{verbatim}
package demo.value;

public class boxedLong
    implements org.omg.CORBA.portable.ValueBase
{
    public int value;
    private static String[] _ids = { boxedLongHelper.id() };

    public boxedLong(int initial )
    {
        value = initial;
    }
    public String[] _truncatable_ids()
    {
        return _ids;
    }
}
\end{verbatim}

The boxed value definitions in IDL above permit uses of non-object
types that are not possible with IDL primitive types. In particular, it
is possible to pass Java {\tt null} references where a value of a
boxed value type is expected. For example, we can call the operation
{\tt receive\_long} and pass one initialized {\tt boxedLong} value and
a null reference, as show in the following snippet from the client
code:

\begin{verbatim}
    ValueServer s = ValueServerHelper.narrow( obj );
    boxedLong boxL = new boxedLong (774);

    System.out.println ("Passing two integers: "
                       + s.receive_long ( boxL , null ));
\end{verbatim}

With a regular {\tt long} parameter, a {\tt null} reference would have
resulted in a {\tt BAD\_PARAM} exception. With boxed value types, this
usage is entirely legal and the result string returned from the {\tt
  ValueServer} object is {\tt ``one or two null values''}.

A second new possibility of the reference semantics that can be
achieved by ``boxing'' primitive IDL types is {\em sharing} of values.
With primitive values, two variables can have copies of the same
value, but they cannot both refer to the same value. This means that
when one of the variables is changed, the other one retains its
original value. With shared values that are {\em referenced}, both
variables would always point to the same value.

The stateful value type {\tt Node} is implemented by the programmer in
a class {\tt NodeImpl} (see the JacORB distribution for the actual
code).  The relationship between this implementation class and the
corresponding IDL definition is not entirely trivial, and we will
discuss it in detail below.

\section{Factories}

When an instance of a (regular) value type is marshaled over the wire
and arrives at a server, a class that implements this value type must
be found, so that a Java object can be created to hold the state
information.  For interface types, which are only passed by
reference, something similar is accomplished by the POA, which
accepts remote calls to the interface and delivers them to a local
implementation class (the \emph{servant}).  For value type instances,
there is no such thing as a POA, because they cannot be called
remotely.  Thus, the ORB needs a different mechanism to know which
Java implementation class corresponds to a given IDL value type.

The CORBA standard introduces \emph{value factories} to achieve this.
Getting your value factories right can be anywhere from trivial to
tricky (we will cover the details in a minute), and so the standard
suggests that ORBs also provide convenience mechanisms to relieve
programmers from writing value factories if possible.  JacORB's
convenience mechanism is straightforward:

\begin{quotation}
\noindent \emph{If the implementation class for an IDL value type A is named
  AImpl, resides in the same package as A, and has a no-argument
constructor, then no value factory is needed for that type.}
\end{quotation}

In other words, if your implementation class follows the common naming
convention (``...Impl''), and it provides a no-arg constructor
so that the ORB can instantiate it, then the ORB has all that it
needs to (a) find the implementation class, and (b) create an instance
of it (which is then initialized with the unmarshaled state from the
wire).

This mechanism ought to save you from having to write a value factory
99\% of the time.  It works for all kinds of regular value types,
including those with inheritance, and recursive types (where a type
has members of its own type).

If you do need more control over the instance creation process, or the
unmarshaling from the wire, you can write your own value factory
class and register it with the ORB using {\tt
ORB.register\_value\_factory(}\emph{repository\_id, factory}{\tt)}.
The \emph{factory} object needs to implement the interface {\tt
org.omg.CORBA.portable.ValueFactory}, which requires a single method:

\begin{verbatim}
   public Serializable read_value (InputStream is);
\end{verbatim}

When an instance of type \emph{repository\_id} arrives over the wire,
the ORB calls the {\tt read\_value()} method, which must unmarshal the
data from the input stream, create an instance of the appropriate
implementation class from it, and return that.

The easiest way to implement this method is to create an instance of
the implementation class, and pass it to the {\tt read\_value()} method
of the given InputStream:

\begin{verbatim}
   public Serializable read_value (InputStream is) {
     A result = new AImpl();
     return is.read_value(result);
   }
\end{verbatim}

The {\tt InputStream.read\_value()} method registers the newly created
instance in the stream's indirection table, and then reads the data
from the stream and initializes the given {\tt value} instance from
it.

The value factory must be registered with the ORB using {\tt
 register\_value\_factory()}.  As a special convenience (defined in
 the CORBA standard), if the value factory class for type {\tt A} is
 called {\tt ADefaultFactory}, then the ORB will find it automatically
 and use it, unless a different factory has been explicitly
 registered.

It sometimes causes confusion that you can also define \emph{factory
methods} in a value type's IDL.  These factory methods are completely
unrelated to the unmarshaling mechanism discussed above; they are
simply a portable means to declare what kinds of ``constructors'' a
value type implementation should have.  They are purely for local use,
but since they are ``factories'', the corresponding methods must also
be implemented in the type's {\tt ValueFactory} implementation.


%%% Local Variables:
%%% mode: latex
%%% TeX-master: "../ProgrammingGuide"
%%% End:
